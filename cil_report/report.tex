\documentclass[10pt,conference,compsocconf]{IEEEtran}

%\usepackage{times}
%\usepackage{balance}
\usepackage{url}
\usepackage{graphicx}	% For figure environment


\begin{document}
\title{A framework for collaborative filters}

\author{
  Yanping Xie, Zuoyue Li, Jie Huang\\
  Department of Computer Science, ETH Zurich, Switzerland
}

\maketitle

\begin{abstract}
  Collaborative filters help people evaluating items based on the opinions of other 
people. In this paper, we propose several different collaborative filter methods and build a common
hybrid framework to combine those techniques for a final recommendation. In traditional collaborative filters, it is usual to 
apply only single algorithm. In our systems, it is flexible for extending and mixing new models, 
which can  quickly produce high quality recommendations. \\
We conduct four methods: regularized singular value decomposition(RSVD), K-means, improved RSVD and Postprocessing SVD with kernel ridge regression. And we use linear model to weight four models. The framework gains a XXX\% improvement comparing to the SVD baseline in the movie dataset.

\end{abstract}

\section{Introduction}

Recommender systems are widely used in nowadays life, which help us choose books, movies, music and even friends. In collaborative filter, the recommendation is based on the preference of the user towards other items and the opinion of others[--- add some reference later---]. In many traditional collaborative filters work, they use only single model which is limited in some specific application. For example, some methods perform quite well in the book recommendation but is weak in the music recommendation[--- add some reference later---]. In this paper, we propose a hybrid framework to mix various recommendation algorithms. \\
The rest of the paper is organized as follows. In the section 2, we first introduce regularized singular value decomposition(RSVD), K-means, improved RSVD and Postprocessing SVD with kernel ridge regression which are used in our system. Then we describe how we combine those four methods. In the section 3, we present our experiment design and some results. In the section 4, we give some conclusions. 

\begin{table*}[htbp]
  \centering
  \begin{tabular}[c]{|l||l|l|l|}
    \hline
    Predictor&RMSE/Linear/SGD&RMSE/Nolinear/FGD&RMSE/linear/FGD\\
    \hline
    SVD&1.007&0.0&0.0\\
    RSVD&1.020&0.0&0.0\\
    RSVD2&1.009&0.0&0.0\\
    K-means&1.046&0.0&0.0\\
    RSVD-with krr&1.068&0.0&0.0\\
    RSVD2-with krr&1.080&0.0&0.0\\
    Linear Model&1.018&0.0&0.0\\
    Merge Model&0.978&0.0&0.0\\
    \hline
  \end{tabular}
    \caption{  \label{mytable} RMSE of different model}
\end{table*}

\section{Models and Methods}
\subsection{Problem Formalization}
In this paper we focus on the movie rating problem. We are given some training samples including user $i$, movie $j$ and their corresponding ratings $r_{ij}$. All ratings are integer values between 1 and 5 stars. And we should evaluate the ratings for new pairs user $i'$ and movie $j'$.

\subsection{Basic Model}
\subsubsection{Regularized singular value decomposition(RSVD)}
We map both users and movies to a joint latent factor space of dimensionality $k$. Accordingly each user $i$ is associated with a vector $u_{i} \in \textbf{R$^k$}$, while each movie $j$ is associated with a vector $v_{j} \in \textbf{R$^k$} $. The dot product $u_i^Tv_j$ captures the interaction between user $i$ and movie $j$. We use this product to estimate the rating of user $i$ and item $j$.

$$ \hat{r_{ij}}=u_{i}^{T}v_{j} $$

To learn the factor of the user and item, the system minimizes the regularized squared error on the set of known ratings.
$$  \min \limits_{u, v} \sum_{i, j\in kn} (r_{ij} - \hat r_{ij})^2 + \lambda (||u_i||^2 + ||v_j||^2)  $$

where $ (||u_i||^2 + ||v_j||^2)$ is the regular terms to avoid overfitting and $kn$ is the set of pair $(i, j)$ of which rating $r_{ij}$ is known.

\subsubsection{Improved regularized singular value decomposition}
In the RSVD methods, we use dot product to fit the real rating. However, it is shown that some biases exist among both users and items. For example, some users tend to give higher rate than others. So we introduce the bias term into the evaluation.

$$ \hat{r_{ij}}=u_{i}^{T}v_{j} + b_i + b_j  + \mu $$
where $b_i$ and $b_j$ are the biases of user and items. $\mu$ is the average of all rating. The system learns by minimizing the squared error function:
$$  \min \limits_{u, v} \sum_{i, j\in kn} (r_{ij} - \hat r_{ij})^2 + \lambda (||u_i||^2 + ||v_j||^2 + b_i^2 + b_j^2)  $$

\subsubsection{K-means}
 K-means algorithm is used to divide users into K clusters $C_k$ and minimizing the intra-cluster variance.
 $$   \sum_{k=1}^K \sum_{i \in C_k} (||r_{i}-\mu_k||^2)  $$, where $$  ||r_{i}-\mu_k||^2 = \sum_{j \in kn_i}(r_{ij}-\mu_{kj})^2  $$, where $kn_i$ is the set of movie rated by user $i$. 
 For each user $i$ in cluster $k$, we use $\mu_{kj}$ to rate movie $j$. Besides we compute the average 11 runs of K-means with $K\in$ [4, 24] and stride 2 as the prediction.

\subsubsection{Postprocessing SVD with kernel ridge regression}
.  \\

[bu shi hen dong a xiong di !]
. \\



\subsection{Improvement}
\subsubsection{Full Gradient vs Stochastic Gradient}
In order to minimize the error of mean square, Stochastic Gradient Descent(SGD) is widely used [------ reference------]. But SGD is easily influenced by noise data, although it owns fast convergence speed. So we implement both full gradient decent and stochastic gradient decent and compare them in the experiment part.

\subsubsection{Linear Interaction vs Nolinear Interaction}
Either RSVD and improved RSVD adopt linear Interaction between user and items. We conduct a nolinear function $f$ in the interaction part.
$$ \hat{r_{ij}}=f(u_{i}^{T}v_{j})$$
$$ \hat{r_{ij}}=f(u_{i}^{T}v_{j}) + b_i + b_j  + \mu $$
where we use $f(x) = 4*sigmod(x)+1 in our experiment$
\subsection{Ensemble}



\section{Experiment}
\subsection{Datasets}
We use a movie rating data set to evaluate variants of item-based recommendation algorithms. The datasets include ratings of 10000 users for 1000 different movies. All ratings are integer values between 1 and 5 stars.The number of known rating is 1176952 and the average is $3.857$. And we split the datasets into two part: 90\% training set and 10\% test set.

\subsection{Metric}
Our collaborative filtering algorithm is evaluated according to the following weighted criteria:
prediction error, measured by root-mean-squared error (RMSE)

\subsection{Experimental Results}
We implement six models: SVD, RSVD, Improved RSVD(shown as RSVD2), RSVD with KRR, RSVD2 with KRR and K-means in the experiment. We compare different training methods(FGD/SGD) and different interaction
function(linear/nolinear) on those six models. Besides, we compare using single model and the ensemble models. All results are summarizes in the table \ref{mytable}.

\subsubsection{Full Gradient vs Stochastic Gradient}
As shown in the second and third column, FGD performs better than SGD. On the model XX, FGD obtain an XXX\% improvement on XXmodel(XXX\%) than SGD. 

\subsubsection{Linear Interaction vs Nolinear Interaction}
The Nolinear Interaction acquire a better results than linear interaction on all models. More specifically, it has a largest improvement on XXmodel(XXX\%), while has limited effect on XX model (XXX\%).

\subsubsection{Single model vs Ensemble model}
The ensemble model that combines all six above-mentioned model achieve a XXX RMSE. The ensemble gain a XXX improvement compare to XXX, which is the best model among all single model. The results demonstrate that the ensemble methods strengthen the ability of model to fit complicated data.


All experiments were done on the ETH cluster euler with XXXGHz processor and XXXGB RAM. Running times varied from XXXXmin for SVD KNN to around XXX for RSVD2.

\section{Summary}
\subsection{Conclusion}
We introduce various basic collaborative filters algorithms and build a framework to combine them all. Besides, we compare different training methods and interaction functions to gain a better movie rating. The experiments show out that an ensemble model with FGD and nolinear interaction model achieve a best results(XXX \%)

\subsection{Future Work} 
\subsubsection{Model} 
We may try to combine some deep learning models in our framework.
\subsubsection{Data} 
Now we just limit our model on movie rating datasets. More experiments are needed on other recommendation tasks to .




\section*{Acknowledgements}
The author thanks Christian Sigg for his careful reading and helpful
suggestions.

\bibliographystyle{IEEEtran}
\bibliography{howto-paper}
\end{document}
